% arara: pdflatex
% arara: pdflatex
\documentclass[12pt]{article}
%%% This file can never be completed.
%%% If you need something but cannot find it,
%%% contact the TA Favonia!

%%%%%%%%%%%%%%%%%%%%%%%%%%%%%%%%%%%%%%%%
% Basic packages
%%%%%%%%%%%%%%%%%%%%%%%%%%%%%%%%%%%%%%%%
\usepackage{amsmath,amsthm,amssymb}
\usepackage{fancyhdr}
\usepackage{mathpartir}
\usepackage{xcolor}
\usepackage{hyperref}
\usepackage{xspace}
\usepackage{comment}
\usepackage{url} % for url in bib entries

%%%%%%%%%%%%%%%%%%%%%%%%%%%%%%%%%%%%%%%%
% Acronyms
%%%%%%%%%%%%%%%%%%%%%%%%%%%%%%%%%%%%%%%%
\usepackage[acronym, shortcuts]{glossaries}

\newacronym{HoTT}{HoTT}{homotopy type theory}
\newacronym{IPL}{IPL}{intuitionistic propositional logic}
\newacronym{TT}{TT}{intuitionistic type theory}
\newacronym{LEM}{LEM}{law of the excluded middle}
\newacronym{ITT}{ITT}{intensional type theory}
\newacronym{ETT}{ETT}{extensional type theory}
\newacronym{NNO}{NNO}{natural numbers object}

%%%%%%%%%%%%%%%%%%%%%%%%%%%%%%%%%%%%%%%%
% Fancy page style
%%%%%%%%%%%%%%%%%%%%%%%%%%%%%%%%%%%%%%%%
\pagestyle{fancy}
\newcommand{\metadata}[2]{
  \lhead{}
  \chead{}
  \rhead{\bfseries Homotopy Type Theory}
  \lfoot{#1}
  \cfoot{#2}
  \rfoot{\thepage}
}
\renewcommand{\headrulewidth}{0.4pt}
\renewcommand{\footrulewidth}{0.4pt}


\newcommand*{\vocab}[1]{\emph{#1}}
\newcommand*{\latin}[1]{\textit{#1}}

%%%%%%%%%%%%%%%%%%%%%%%%%%%%%%%%%%%%%%%%
% Customize list enviroonments
%%%%%%%%%%%%%%%%%%%%%%%%%%%%%%%%%%%%%%%%
% package to customize three basic list environments: enumerate, itemize and description.
\usepackage{enumitem}
\setitemize{noitemsep, topsep=0pt, leftmargin=*}
\setenumerate{noitemsep, topsep=0pt, leftmargin=*}
\setdescription{noitemsep, topsep=0pt, leftmargin=*}

%%%%%%%%%%%%%%%%%%%%%%%%%%%%%%%%%%%%%%%%
% Some really basic macros.
% (Lots of them were stolen from HoTT/Book.)
% See macros.tex in HoTT/book.
%%%%%%%%%%%%%%%%%%%%%%%%%%%%%%%%%%%%%%%%
\newcommand*{\ctx}{\Gamma}
\newcommand{\entails}{\vdash}


\newcommand*{\judgmentfont}[1]{{\normalfont\sffamily #1}}
\newcommand*{\postfixjudgment}[1]{%
  \relax\ifnum\lastnodetype>0\mskip\medmuskip\fi
  \text{\judgmentfont{#1}}%
}
\newcommand*{\prop}{\postfixjudgment{prop}}
\newcommand*{\true}{\postfixjudgment{true}}
\newcommand*{\type}{\postfixjudgment{type}}
\newcommand*{\context}{\postfixjudgment{ctx}}


\newcommand*{\truth}{\top}
\newcommand*{\conj}{\wedge}
\newcommand*{\disj}{\vee}
\newcommand*{\falsehood}{\bot}
\newcommand*{\imp}{\supset}


%%% Judgmental equality
\newcommand{\jdeq}{\equiv}
%%% Definition
\newcommand{\defeq}{\vcentcolon\equiv}
%%% Binary sums
\newcommand{\inlsym}{{\mathsf{inl}}}
\newcommand{\inrsym}{{\mathsf{inr}}}
\newcommand{\inl}{\ensuremath\inlsym\xspace}
\newcommand{\inr}{\ensuremath\inrsym\xspace}
%%% Booleans
\newcommand{\ttsym}{{\mathsf{tt}}}
\newcommand{\ffsym}{{\mathsf{ff}}}
%\newcommand{\ttrue}{\ensuremath\ttsym\xspace}
%\newcommand{\ffalse}{\ensuremath\ffsym\xspace}
%%% Pairs
\newcommand{\pair}{\ensuremath{\mathsf{pair}}\xspace}
\newcommand{\tuple}[2]{(#1,#2)}
\newcommand{\proj}[1]{\ensuremath{\mathsf{pr}_{#1}}\xspace}
\newcommand{\fst}{\ensuremath{\proj1}\xspace}
\newcommand{\snd}{\ensuremath{\proj2}\xspace}
%%% Path concatenation
\newcommand{\concat}{%
  \mathchoice{\mathbin{\raisebox{0.5ex}{$\displaystyle\centerdot$}}}%
  {\mathbin{\raisebox{0.5ex}{$\centerdot$}}}%
  {\mathbin{\raisebox{0.25ex}{$\scriptstyle\,\centerdot\,$}}}%
  {\mathbin{\raisebox{0.1ex}{$\scriptscriptstyle\,\centerdot\,$}}}
}
%%% Transport (covariant)
\newcommand{\trans}[2]{\ensuremath{{#1}_{*}\mathopen{}\left({#2}\right)\mathclose{}}\xspace}
% Natural numbers objects
\newcommand{\Nat}{\mathsf{Nat}}
\newcommand{\rec}{\ensuremath{\mathsf{rec}}\xspace}
% Sequence
\newcommand{\Seq}{\ensuremath{\mathsf{Seq}}\xspace}
% Identity type
\newcommand{\Id}[1]{\ensuremath{\mathsf{Id}_{#1}}\xspace}
% Reflection
\newcommand{\refl}[1]{\ensuremath{\mathsf{refl}_{#1}}\xspace}

% fst,snd,case,id
\renewcommand*{\fst}{\textsf{fst}}
\renewcommand*{\snd}{\textsf{snd}}
\DeclareMathOperator{\case}{\textsf{case}}
\DeclareMathOperator{\caseif}{\textsf{if}}
\DeclareMathOperator{\casesplit}{\textsf{split}}
\DeclareMathOperator{\ttrue}{\textsf{tt}\xspace}
\DeclareMathOperator{\ffalse}{\textsf{ff}\xspace}
\newcommand*{\id}{\textsf{id}}

\metadata{Vonnie III}{2099/99/99}

\usepackage{proof}
\usepackage{glossaries}

\begin{document}

\section{Intuitionistic Propositional Logic: ``Logic as if people matter''}\label{sec:ipl}

\newacronym{IPL}{IPL}{intuitionistic propositional logic}%
%
As advanced by Per Martin-L\"{o}f, a modern presentation of \gls{IPL} distinguishes the notions of \vocab{judgement} and \vocab{proposition}.
A judgment is something that may be known, whereas a proposition is something that sensibly be may the subject of a judgment.
For instance, 

% Ultimately, we are interested in judging the truth, or provability, of a proposition.
% But first, 

Thus, in \gls{IPL}, the two most basic judgements are $A \prop$ and $A \true$:


\subsection{Negative fragment of \gls{IPL}}\label{sec:ipl-negative}

\subsubsection{Truth}\label{sec:truth}

One of the simplest propositions is \vocab{truth}, which we write as $\truth$.
Its formation rule serves as immediate evidence of the judgment $\truth \prop$, that $\truth$ is indeed a well-formed proposition.
\begin{equation*}
  \infer[{\truth}F]{\truth \prop}{
    }
\end{equation*}

We have yet to give meaning to truth, however; to do so, we must say what counts as a verification of $\truth$.
The proposition $\truth$ is trivially true, and so its introduction rule makes the judgment $\truth \true$ immediately evident.
\begin{equation*}
  \infer[{\truth}I]{\truth \true}{
    }
\end{equation*}

We should also pause to consider an elimination rule for $\truth$: from a proof of $\truth \true$, what can we deduce?
Since $\truth$ is trivially true, any such elimination rule would not increase our knowledge---we put in no information when we introduced $\truth \true$, so, by the principle of conservation of proof, we should get no information out.
For this reason, there is no elimination rule for $\truth$.

To summarize, here are the formation, introduction, and elimination rules for truth:
\begin{mathpar}
  \infer[{\truth}F]{\truth \prop}{
    }
  \\
  \infer[{\truth}I]{\truth \true}{
    }
  \and
  \text{(no ${\truth}E$ rule)}
\end{mathpar}

\subsubsection{Conjunction}\label{sec:conjunction}

Another familiar group of propositions are the conjunctions.
If $A$ and $B$ are propositions, then so is their conjunction, which we write as $A \conj B$.
This is the content of the formation rule for conjunction.
\begin{equation*}
  \infer[{\conj}F]{A \conj B \prop}{
    A \prop & B \prop}
\end{equation*}

To give meaning to conjunction, we must say what counts as a verification of $A \conj B$.
This is done by the following introduction rule, which shows that a verification of $A \conj B$ consists of a verification of $A$ paired with a verification of $B$.
\begin{equation*}
  \infer[{\conj}I]{A \conj B \true}{
    A \true & B \true}
\end{equation*}

On the flip side, what may we deduce from the knowledge that $A \conj B$ is true?
Because is a pair of verifications of $A$ and $B$, we are justified in deducing $A \true$ and $B \true$.
\begin{mathpar}
  \infer[{\conj}E_1]{A \true}{
    A \conj B \true}
  \and
  \infer[{\conj}E_2]{B \true}{
    A \conj B \true}
\end{mathpar}

\subsubsection{Logical entailment}\label{sec:logical-entailment}

\subsubsection{Implication}\label{sec:implication}

\subsubsection{Summary of the negative fragment of \gls{IPL}}\label{sec:summary-negative}

\end{document}
