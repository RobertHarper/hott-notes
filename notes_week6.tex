\documentclass[12pt]{article}
\usepackage[letterpaper]{geometry}                                              
                                                                                
%%% This file can never be completed.
%%% If you need something but cannot find it,
%%% contact the TA Favonia!

%%%%%%%%%%%%%%%%%%%%%%%%%%%%%%%%%%%%%%%%
% Basic packages
%%%%%%%%%%%%%%%%%%%%%%%%%%%%%%%%%%%%%%%%
\usepackage{amsmath,amsthm,amssymb}
\usepackage{fancyhdr}
\usepackage{mathpartir}
\usepackage{xcolor}
\usepackage{hyperref}
\usepackage{xspace}
\usepackage{comment}
\usepackage{url} % for url in bib entries

%%%%%%%%%%%%%%%%%%%%%%%%%%%%%%%%%%%%%%%%
% Acronyms
%%%%%%%%%%%%%%%%%%%%%%%%%%%%%%%%%%%%%%%%
\usepackage[acronym, shortcuts]{glossaries}

\newacronym{HoTT}{HoTT}{homotopy type theory}
\newacronym{IPL}{IPL}{intuitionistic propositional logic}
\newacronym{TT}{TT}{intuitionistic type theory}
\newacronym{LEM}{LEM}{law of the excluded middle}
\newacronym{ITT}{ITT}{intensional type theory}
\newacronym{ETT}{ETT}{extensional type theory}
\newacronym{NNO}{NNO}{natural numbers object}

%%%%%%%%%%%%%%%%%%%%%%%%%%%%%%%%%%%%%%%%
% Fancy page style
%%%%%%%%%%%%%%%%%%%%%%%%%%%%%%%%%%%%%%%%
\pagestyle{fancy}
\newcommand{\metadata}[2]{
  \lhead{}
  \chead{}
  \rhead{\bfseries Homotopy Type Theory}
  \lfoot{#1}
  \cfoot{#2}
  \rfoot{\thepage}
}
\renewcommand{\headrulewidth}{0.4pt}
\renewcommand{\footrulewidth}{0.4pt}


\newcommand*{\vocab}[1]{\emph{#1}}
\newcommand*{\latin}[1]{\textit{#1}}

%%%%%%%%%%%%%%%%%%%%%%%%%%%%%%%%%%%%%%%%
% Customize list enviroonments
%%%%%%%%%%%%%%%%%%%%%%%%%%%%%%%%%%%%%%%%
% package to customize three basic list environments: enumerate, itemize and description.
\usepackage{enumitem}
\setitemize{noitemsep, topsep=0pt, leftmargin=*}
\setenumerate{noitemsep, topsep=0pt, leftmargin=*}
\setdescription{noitemsep, topsep=0pt, leftmargin=*}

%%%%%%%%%%%%%%%%%%%%%%%%%%%%%%%%%%%%%%%%
% Some really basic macros.
% (Lots of them were stolen from HoTT/Book.)
% See macros.tex in HoTT/book.
%%%%%%%%%%%%%%%%%%%%%%%%%%%%%%%%%%%%%%%%
\newcommand*{\ctx}{\Gamma}
\newcommand{\entails}{\vdash}


\newcommand*{\judgmentfont}[1]{{\normalfont\sffamily #1}}
\newcommand*{\postfixjudgment}[1]{%
  \relax\ifnum\lastnodetype>0\mskip\medmuskip\fi
  \text{\judgmentfont{#1}}%
}
\newcommand*{\prop}{\postfixjudgment{prop}}
\newcommand*{\true}{\postfixjudgment{true}}
\newcommand*{\type}{\postfixjudgment{type}}
\newcommand*{\context}{\postfixjudgment{ctx}}


\newcommand*{\truth}{\top}
\newcommand*{\conj}{\wedge}
\newcommand*{\disj}{\vee}
\newcommand*{\falsehood}{\bot}
\newcommand*{\imp}{\supset}


%%% Judgmental equality
\newcommand{\jdeq}{\equiv}
%%% Definition
\newcommand{\defeq}{\vcentcolon\equiv}
%%% Binary sums
\newcommand{\inlsym}{{\mathsf{inl}}}
\newcommand{\inrsym}{{\mathsf{inr}}}
\newcommand{\inl}{\ensuremath\inlsym\xspace}
\newcommand{\inr}{\ensuremath\inrsym\xspace}
%%% Booleans
\newcommand{\ttsym}{{\mathsf{tt}}}
\newcommand{\ffsym}{{\mathsf{ff}}}
%\newcommand{\ttrue}{\ensuremath\ttsym\xspace}
%\newcommand{\ffalse}{\ensuremath\ffsym\xspace}
%%% Pairs
\newcommand{\pair}{\ensuremath{\mathsf{pair}}\xspace}
\newcommand{\tuple}[2]{(#1,#2)}
\newcommand{\proj}[1]{\ensuremath{\mathsf{pr}_{#1}}\xspace}
\newcommand{\fst}{\ensuremath{\proj1}\xspace}
\newcommand{\snd}{\ensuremath{\proj2}\xspace}
%%% Path concatenation
\newcommand{\concat}{%
  \mathchoice{\mathbin{\raisebox{0.5ex}{$\displaystyle\centerdot$}}}%
  {\mathbin{\raisebox{0.5ex}{$\centerdot$}}}%
  {\mathbin{\raisebox{0.25ex}{$\scriptstyle\,\centerdot\,$}}}%
  {\mathbin{\raisebox{0.1ex}{$\scriptscriptstyle\,\centerdot\,$}}}
}
%%% Transport (covariant)
\newcommand{\trans}[2]{\ensuremath{{#1}_{*}\mathopen{}\left({#2}\right)\mathclose{}}\xspace}
% Natural numbers objects
\newcommand{\Nat}{\mathsf{Nat}}
\newcommand{\rec}{\ensuremath{\mathsf{rec}}\xspace}
% Sequence
\newcommand{\Seq}{\ensuremath{\mathsf{Seq}}\xspace}
% Identity type
\newcommand{\Id}[1]{\ensuremath{\mathsf{Id}_{#1}}\xspace}
% Reflection
\newcommand{\refl}[1]{\ensuremath{\mathsf{refl}_{#1}}\xspace}

% fst,snd,case,id
\renewcommand*{\fst}{\textsf{fst}}
\renewcommand*{\snd}{\textsf{snd}}
\DeclareMathOperator{\case}{\textsf{case}}
\DeclareMathOperator{\caseif}{\textsf{if}}
\DeclareMathOperator{\casesplit}{\textsf{split}}
\DeclareMathOperator{\ttrue}{\textsf{tt}\xspace}
\DeclareMathOperator{\ffalse}{\textsf{ff}\xspace}
\newcommand*{\id}{\textsf{id}}
                                                                  
                                                                                
\usepackage{proof-dashed}                                                       
\usepackage{tikz-cd}                                                            
\usepackage{amsmath}                                                            
\usepackage{lmodern}                                                            
\usepackage{microtype}                                                          

\metadata{Robert Lewis and Joseph Tassarotti}{2013/10/21 and 2013/10/23}

\newtheorem{thm}{Theorem}
\newtheorem{eg}{Example}
\newcommand{\ap}{\mathsf{ap}}
\newcommand{\apd}{\mathsf{apd}}
\newcommand{\tr}{\mathsf{tr}}
\newtheorem*{remark}{Remark}

\begin{document}
\title{15-819 Homotopy Type Theory Lecture Notes} 
\author{Robert Lewis and Joseph Tassarotti}
\date{October 21 and 23, 2013}

\maketitle

\section{Paths-over-Paths}\label{}

Recall that last time we explored the higher groupoid structure of types, and
showed that for non-dependent maps, $\ap$ preserves this structure.  Now, in
the case where we have a dependent function $f : \Pi x: A.B$, we would like to
similarly state that $f$ maps equals to equals, so that given a path $p :
\Id{A}(M,N)$, there is some map which takes $p$ gives a path between $f M$ and
$f N$. However, because $f$ is dependent, $f M : [M/x]B$ and $f N : [N/x] B$.
Although these types are related, they are not equal, so we cannot talk about
propositional equality between $f M$ and $f N$.

In earlier lectures, we defined $\tr[x.B]p : [M/x] B \to [N/x] B$, often
written as $p_*$, which lifts the path $p$ to a mapping between the fibers
$[M/x] B$ and $[N/x] B$. Since $p_*(f M)$ and $f N$ share the same type, we can
meaningfully talk about equality between them. We can now define a map $\apd_f
: \Pi p: \Id{A}(M, N). \Id{[N/x]B}(p_* (f M), f N)$ by
%
\[ \apd_f p := \J[m.n.z. \Id{[n/x]B}(z_{*}(f m), f n)](p ; m. \refl{[m/x]B}(m)) \]
%

This has the appropriate type because when the path is simply $\refl{A}(M)$, we
have that $(\refl{A}(M))_* \jdeq \refl{[M/x]B}(f M)$ Now, since $p_*^{-1}$
gives a map between the fibers going the other way, we could just as well have
defined an analogous term $\apd_f' : Id{A}(M, N) \to \Id{[M/x]B}(f M,
p_*^{-1}(f N))$. Moreover, we have that
%
\[
\begin{array}{lcl}
\ap_{p_*^{-1}} (\apd_f p) &:& \Id{[M/x]B}(p_*^{-1} (p_* (f M)), p_*^{-1} (f N))  \\
                          & &  \ \ \jdeq \Id{[M/x]B}(f M, p_*^{-1} (f N)) \\
\\
\ap_{p_*} (\apd_f p^{-1}) &:& \Id{[N/x]B}(p_* (f M)), p_* (p_*^{-1} (f N))) \\
                          & &  \ \ \jdeq \Id{[N/x]B}(p_* (f M), f N) \\

\end{array}
\]
%
which shows that these two theorems imply one another. 

The lack of symmetry in the types of $\apd_f$ and $\apd_f'$ is somewhat awkward
when developing machine checked proofs. It's more convenient to define a
symmetric notation, $f(M) =_p^{x. B} f(N) \defeq Id{[N/x]B}(p_* (f M), f N)$,
which we read as ``$f(M)$ and $f(N)$ are correlated by $p$" . This corresponds
to the type of paths \emph{over} the path $p$. Using this notation, we 
can prove theorems about this type like:


\[
\begin{array}{lcl}
\text{sym}_{\text{corr}} & : & Q =_p^{x.B} R \ \to \ R =_{p^{-1}}^{x.B} Q \\
\\
\text{trans}_{\text{corr}} & : & Q =_p^{x.B} R \ \to R =_q^{x.B} S \to Q =_{p \concat q}^{x.B} S 
\end{array}
\]
%
[TODO : Draw diagrams? Give terms for the correlation lemmas?]

\section{Equivalence}

\bibliographystyle{plain}
\bibliography{hott_references}

\end{document}
